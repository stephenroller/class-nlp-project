\documentclass{article}

\title{CS388 Final Project Writeup:\\ 
Unknown Noun Supersense Acquisition using Search Engines}

\author{Stephen Roller \& Karl Pichotta}


\begin{document}

\maketitle




\section{Introduction}

Motivate and abstractly describe the problem you are addressing and how you are addressing it. What is the problem? Why is it important? What is your basic approach? A short discussion of how it fits into related work in the area is also desirable. Summarize the basic results and conclusions that you will present. 

\section{Problem Definition and Algorithm}

\subsection{Task Definition}

Precisely define the problem you are addressing (i.e. formally specify the inputs and outputs). Elaborate on why this is an interesting and important problem. 

\subsection{Algorithm Definition}

Describe in reasonable detail the algorithm you are using to address this problem. A psuedocode description of the algorithm you are using is frequently useful. Trace through a concrete example, showing how your algorithm processes this example. The example should be complex enough to illustrate all of the important aspects of the problem but simple enough to be easily understood. If possible, an intuitively meaningful example is better than one with meaningless symbols. 

\section{Experimental Evaluation}

\subsection{Methodology}

What are criteria you are using to evaluate your method? What specific hypotheses does your experiment test? Describe the experimental methodology that you used. What are the dependent and independent variables? What is the training/test data that was used, and why is it realistic or interesting? Exactly what performance data did you collect and how are you presenting and analyzing it? Comparisons to competing methods that address the same problem are particularly useful. 

\subsection{Results}

\begin{center}
\begin{figure}[hbtp]
\begin{tabular}{|l| l| l| l|}
\hline
{\bf setup} & {\bf Precision} & {\bf Recall} & {\bf F1-score}  \\
\hline \hline
most common superclass baseline & 20.00 & 20.00 & 20.00 \\
\hline
No initial enrichment, No refinement & 33.33 & 30.00 & 31.58 \\
\hline
No init, refinement with webpages & 39.51 & 32.00 & 35.36 \\
\hline
No init, refinement with descriptions & 38.27 & 31.00 & 34.25 \\
\hline
Initial enrichment, No refinement & 33.33 & 30.00 & 31.58 \\
\hline
init, refinement with webpages & 39.51 & 32 & 35.36 \\
\hline
No init, refinement with descriptions & 38.27 & 31.00 & 34.25 \\
\hline
\end{tabular}
\caption{Results on {\tt small-testset}}
\label{fig:res1}
\end{figure}
\end{center}

Present the quantitative results of your experiments. Graphical data presentation such as graphs and histograms are frequently better than tables. What are the basic differences revealed in the data. Are they statistically significant? 

\subsection{Discussion}

Is your hypothesis supported? What conclusions do the results support about the strengths and weaknesses of your method compared to other methods? How can the results be explained in terms of the underlying properties of the algorithm and/or the data. 

\section{Related Work}

Answer the following questions for each piece of related work that addresses the same or a similar problem. What is their problem and method? How is your problem and method different? Why is your problem and method better? 

\section{Future Work}

What are the major shortcomings of your current method? For each shortcoming, propose additions or enhancements that would help overcome it. 

\section{Conclusion}

Briefly summarize the important results and conclusions presented in the paper. What are the most important points illustrated by your work? How will your results improve future research and applications in the area? 




\bibliography{writeup}{}
\bibliographystyle{plain}


\end{document}